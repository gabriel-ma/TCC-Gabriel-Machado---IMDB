\subsubsection{WEKA}

O WEKA, \textit{Waikato Environment for Knowledge Analysis}, segundo \citeAuthorPageYear{weka}, é uma coleção de algoritmos de aprendizado de máquina e de ferramentas de preprocessamento de dados. Ele é criado de uma forma que os algoritmos possam ser testados rapidamente nos \textit{datasets}. Também contém suporte de mineração de dados, como preparação dos dados, avaliar modelos estatisticamente, visualização dos dados e resultados da aprendizagem. Todas essas ferramentas estão disponíveis em uma interface simples. O WEKA é escrito em Java e é distribuído sob a GNU \textit{(General Public License)}, rodando em quase toda plataforma e já tendo sido testado nos sistemas operacionais Linux, Windows e Macintosh.

\citeAuthorPageYear{weka} afirmam que o WEKA contém métodos para os principais problemas de mineração de dados, como regressão, clusterização, mineração de regras de associação e seleção de atributos. Conhecer os dados é uma parte integral do trabalho. Uma das formas de se usar o WEKA é aplicando um método aos dados e analisando a saída para aprender mais sobre os dados, outra é selecionando modelos para gerar previsões em novas instâncias. Uma terceira forma é aplicar diferentes algoritmos e comparar as performances para então selecionar um para previsão. A implementação de algoritmos de aprendizado é um dos recursos mais valiosos que o WEKA fornece, enquanto os métodos de preprocessamento, também chamados de \textit{filters}, vêm logo após. 

Nas palavras de \citeAuthorPageYear{weka}, o WEKA é facilmente usado através de uma interface gráfica chamada de \textit{Explorer}. Por exemplo, um \textit{dataset} é facilmente carregado e construir uma árvore de decisão em cima dele. O WEKA também tem outras interfaces, como o \textit{Knowledge Flow}, que facilita processar dados via \textit{streaming} e o \textit{Experimenter}, que permite responder uma pergunta bem básica e prática: quais métodos e valores de parâmetros funcionam melhor para determinado problema? O WEKA fornecer um ambiente que possibilita aos usuários do WEKA comparar uma variedade de algoritmos de aprendizado. É possível fazer isso interativamente usando a  interface \textit{Explorer}, mas o \textit{Experimenter} facilita a automatização do processo, deixando mais fácil de rodar classificadores e filtros em um \textit{dataset}, para coletar estatísticas de performance.

Além disso, todas essas opções podem ser acessadas através de uma interface de linha de comando.