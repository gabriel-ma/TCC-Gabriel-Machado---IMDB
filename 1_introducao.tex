\section{INTRODUÇÃO}

Existem diversas fontes de dados que podem conter algum tipo de conhecimento que não está visível aos nossos olhos, um certo padrão que não é tão fácil reconhecer sem alguma computação. A mineração de dados pode reconhecê-los e nos ajudar a decidir o que fazer com eles, como deixar um produto mais próximo de outro em um supermercado, pois um algoritmo de mineração descobriu que esses dois produtos são geralmente comprados em conjunto.

A mineração de dados pode estar ligada à inteligência artificial, já que métodos de aprendizado de máquina podem ser utilizados para encontrar padrões e nos auxiliar a descrever melhor o que foi encontrado.
Pensando nisso, o objetivo deste trabalho é utilizar a mineração de dados em uma base do IMDB (\textit{Internet Movie Database}) para tentar encontrar padrões que possam nos ajudar a prever a nota de filmes do IMDB.

Vários fatores podem levar ao fracasso ou ao sucesso de um filme, como não escolher bem os diretores, os atores, pouco investimento etc. O IMDB contém informações de filmes, séries e animações, e para cada um desses existe uma nota relacionada. Essa nota é atribuída pelos usuários e é geralmente levada em consideração ao se escolher um filme, afinal, não é tão comum alguém querer assistir um filme que tem uma nota muito baixa.

A possibilidade de prever essa avaliação dos usuários seria uma ferramenta útil durante a produção de um novo filme, os estúdios não gostariam que os seus filmes tenham péssimas avaliações e já podem trabalhar em cima do que foi previsto e possivelmente melhorar a nota.

Para isso, a mineração de dados irá se basear em dados históricos dos filmes para buscar encontrar algo que possa ajudar a melhorar a nota de um filme. Esse algo pode ser uma diretora especifica, línguas em que o filme foi lançado etc.

Visto isso, algumas hipóteses foram levantadas e os modelos foram usados para validá-las (ou não).

Para prever a nota de um filme baseado em dados históricos, aqui será discutido sobre a criação de Data Warehouses, que utiliza o processo de ETL (\textit{Extract, Transform, Load}), que será aplicado nesse trabalho, para serem criados. CRISP-DM, metodologia usada na mineração de dados e com uma visão de Bussiness Intelligence e uma ferramenta para auxiliar o ETL, o \pdi, junto com a linguagem de programação \textit{Python}.

Aqui também terá uma pequena introdução ao WEKA (\textit{Waikato Environment for Knowledge Analysis}), uma poderosa ferramenta para mineração de dados, com diversas formas de aplicar algoritmos em cima dos dados construídos.

Nesse capítulo, a motivação e os objetivos do trabalho foram apresentados. No Capítulo 2, uma visão geral sobre Data Warehouse será mostrada. No Capítulo 3, as principais características da plataforma Pentaho Data Integration serão introduzidas. Um breve resumo sobre o mineração de dados e a ferramenta WEKA é mostrado no Capítulo 4. O Capítulo 5 apresenta a metodologia utilizada no estudo de caso e também exibe e discute os principais resultados obtidos. A conclusão desse trabalho é apresentada no Capítulo 6.