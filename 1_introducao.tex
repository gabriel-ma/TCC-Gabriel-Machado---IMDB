\section{INTRODUÇÃO}

Existem diversas fontes de dados que podem conter algum tipo de conhecimento que não está visível aos nossos olhos, algum padrão que não é tão fácil reconhecer sem algum tipo de computação. A mineração de dados pode reconhece-los e nos ajudar a decidir o que fazer com eles, como deixar um produto mais próximo de outro em um supermercado pois um algoritmo de mineração descobriu que esses dois produtos são geralmente comprados em conjunto.

A mineração de dados está pode estar ligada a inteligência artificial, já que métodos de aprendizado de máquina podem ser utilizados para encontrar padrões e nos ajudar a descrever melhor o que foi encontrado.
Pensando nisso, o objetivo desse trabalho é utilizar mineração de dados em uma base de dados do IMDB (Internet Movie Database) para tentar encontrar algum ou alguns padrões que possam nos ajudar a prever a nota de filmes do IMDB.

Vários fatores podem levar ao fracasso ou ao sucesso de um filme, como não escolher bem os diretores, os atores, pouco investimento e etc. O IMDB contém informações de filmes, séries, animações e para cada um desses existe uma nota relacionada. Essa nota é atribuída pelos usuários e é geralmente levada em consideração ao se escolher um filme, afinal, não é muito comum decidir assistir um filme que tem uma nota muito baixa.

Prever essa avaliação dos usuários pode ser muito útil durante a produção de um novo filme, os estúdios não gostariam que os seus filmes tenham péssimas avaliações e já podem trabalhar em cima do que foi previsto e tentar melhorar a nota.

Para isso, a mineração de dados irá se basear em dados históricos dos filmes para tentar encontrar algo que possa ajudar a melhorar a nota de um filme. Esse algo pode ser uma diretora especifica, línguas em que o filme foi lançado e etc.

Visto isso, alguma hipóteses foram levantadas e os modelos foram usados para validar ou não elas.

Para tentar prever a nota de um filme baseado em dados históricos, aqui será discutido sobre a criação de Data Warehouses, que utiliza o processo de ETL, que será utilizado nesse trabalho, para serem criados. CRISP-DM, metodologia utilizada na mineração de dados e com uma visão de Bussiness Intelligence e a ferramenta utilizada para o auxiliar o ETL, o \pdi, junto com a linguagem de programação \textit{python}.

Aqui também terá uma pequena introdução ao WEKA, que é uma poderosa ferramenta para mineração de dados, com diversas formas de aplicar algoritmos em cima dos dados construídos.

Nesse capítulo, a motivação e os objetivos do trabalho foram apresentados. No Capítulo 2, oferecemos uma visão geral sobre Data Warehouse. No Capítulo 3, introduzimos as principais características da plataforma Pentaho Data Integration. Um breve resumo sobre o mineração de dados e a ferramenta WEKA é mostrado no Capítulo 4. O Capítulo 5 apresenta a metodologia utilizada no estudo de caso. O Capítulo 6 apresenta e discute os principais resultados obtidos. A conclusão desse trabalho é apresentada no Capítulo 7.