\section{CONCLUSÕES}
À luz dos resultados encontrados, é possível concluir que prever as notas de um filme é um trabalho um tanto complicado. Diversas observações e transformações precisam ser realizadas para se ter um resultado apropriado. Além de que uma maior quantidade de amostras seria ideal também.

Diante disso, o trabalho aqui realizado foi capaz de prever as notas com um erro relativamente baixo, de aproximadamente 0.7245, usando \textit{percentage split}. Foi difícil saber com certeza se outros atributos tinham uma influência direta no resultado final, visto que a variação no resultado era muito pouca. Removendo individualmente os atributos de país, diretor e orçamento, o RMSE final variou muito pouco, fazendo com que qualquer conclusão seja imprecisa.

Não foi possível definir se os dados estavam ou não enviesados sem uma observação maior deles, já que o erro, no caso da análise sem os filmes feitos nos Estados Unidos, foi menor. Além de que, sem as produções feitas nos EUA, a quantidade de registros cai de 4998 para 1224, perdendo muitos dados que são de grande importância para a análise de regressão.

Um sistema de recomendações utilizando os resultados desse trabalho é uma possível extensão desse projeto. O sistema poderia levar em consideração a nota que o usuário deu para certos filmes e os gêneros deles, podendo indicar um com uma chance maior de acerto, ou seja, de agradar o usuário.

Além disso, o sistema poderia ajudar a escolher atrizes que têm uma maior influência nas notas, evitando fazer uma escolha possivelmente ruim.