\section{CONCLUSÕES}
Analisando os resultados encontrados, é possível concluir que prever as notas de um filme é um trabalho um tanto complicado. Diversas observações e transformações precisam ser realizadas para ter um resultado apropriado. Além de uma maior quantidade de amostras seria ideal também.

Diante disso, o trabalho aqui realizado foi capaz de prever as notas com um erro relativamente baixo, de aproximadamente 0.7245, usando percentage split. Foi difícil saber com certeza se outros atributos tinham uma influencia direta no resultado final, visto que a variação no resultado era muito pouca. Removendo individualmente os atributos de país, diretor e orçamento, o RMSE final variou muito pouco, fazendo com que qualquer conclusão seja imprecisa.

Não foi possível definir se os dados estavam ou não enviesados sem uma observação maior dos dados, já que o erro, no caso da analise sem os filmes feitos nos Estados Unidos, foi menor. Além de que, sem os filmes feito nos EUA, a quantidade de dados cai de 4998 para 1224, perdendo muitos dados que são de grande importância para a analise de regressão.

Um sistema de recomendações utilizando os resultados desse trabalho é uma possível extensão desse projeto. O sistema poderia levar em consideração a nota que o usuário deu para certos filmes e os gêneros dos filmes, podendo indicar um filme com uma chance maior de acerto, ou seja, de agradar o usuário.

Além disso, o sistema poderia ajudar a escolher atrizes que tem uma maior influência nas notas, evitando fazer uma escolha possivelmente ruim.