\section{CONCLUSÕES}
Neste trabalho, foi explicado o que é um \textit{Data Warehouse}, os tipos, tipos de operação. Também foi explicado o que é o processo de ETL. Uma das ferramentas que auxiliou no processo de ETL \pdi foi apresentada. Outra ferramenta apresentada foi o WEKA, que auxilio no processo de aprendizado de máquina. Foram também apresentados conceitos de CRISP-DM e \textit{Data Mining}.

Em seguida foi apresentado o estudo de caso, onde foi feita uma análise exploratória dos dados, foi mostrado como o processo de ETL foi aplicado, quais hipóteses foram levantadas e os resultados dessas hipóteses. 

A partir delas, foi possível prever a nota de filmes com um RMSE relativamente pequeno, porém que pode ainda ser melhorado. Uma das formas de reduzir ainda mais, é gerando mais atributos derivados. Também não foi possível afirmar que os dados estavam enviesados.

Um sistema de recomendações utilizando os resultados desse trabalho é uma possível extensão desse projeto. O sistema poderia levar em consideração a nota que o usuário deu para certos filmes e os gêneros deles, podendo indicar um com uma chance maior de acerto, ou seja, de agradar o usuário. Além disso, o sistema poderia ajudar a escolher pessoas atrizes que têm uma maior influência nas notas, evitando fazer uma escolha possivelmente ruim.